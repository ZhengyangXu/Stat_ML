

\title{HW0 Stat W4400}
\author{Ethan Grant       uni: erg2145}
\date{\today}

\documentclass{article}
\usepackage{amsmath}
\begin{document}
	\maketitle1
	\begin{enumerate}
		\item $B_{2,1} = 3$
		
		\item $A + B = \begin{bmatrix} 1 & 2 \\ 2 & 4 \end{bmatrix} + \begin{bmatrix} 1 & 2 \\ 3 & 4 \end{bmatrix} = \begin{bmatrix} 2 & 4 \\ 5 & 8 \end{bmatrix}$ 
		\item $A*B = \begin{bmatrix} 1 & 2 \\ 2 & 4 \end{bmatrix} * \begin{bmatrix} 1 & 2 \\ 3 & 4 \end{bmatrix} = \begin{bmatrix} 7 & 10 \\ 14 & 20 \end{bmatrix}$
		\item rank(A) = number of leading 1s in rref(A)\newline rref(A) = $\begin{bmatrix}
		1 & 2 \\ 0 & 0
		\end{bmatrix}$ \t
		rank(A) = 1
		\item $\begin{bmatrix}
		1-\lambda & 2 \\ 2 & 4-\lambda
		\end{bmatrix}
		$
		\newline compute the determinant \newline
		$det(\begin{bmatrix}
			1-\lambda & 2 \\ 2 & 4-\lambda
		\end{bmatrix}) = (1-\lambda)(4-\lambda) = 0$
		\newline
		$4 - \lambda - 4*\lambda + \lambda^{2}-4 = 0$ \newline
		$\lambda^{2} -5*\lambda = 0$ \newline
		$\lambda*(\lambda-5) = 0$ \newline
		$\lambda_{1} = 0$  and  $\lambda_{2} = 5 $ Thus the greatest eigenvalue is 5
		
		
		\item $ker(\begin{bmatrix}
		1 & -1/2 \\ 2 & -1
		\end{bmatrix})$ => $\begin{bmatrix}
		1 & -1/2 \\ 0 & 0 
		\end{bmatrix}$
		The eigenvector is thus $\begin{bmatrix}
		1 \\ 2
		\end{bmatrix}$
		
		
		\item $|B| = (4) - 6 = -2$
		
		
		\item $x^{T}Ax = \begin{bmatrix}
		2 & 1
		\end{bmatrix} * \begin{bmatrix} 1 & 2 \\ 2 & 4 \end{bmatrix} * \begin{bmatrix}
			2 \\ 1
		\end{bmatrix} = \begin{bmatrix}
		2 & 1
		\end{bmatrix} * \begin{bmatrix}
		4 \\ 8
		\end{bmatrix} =16 $
		
		\item $\begin{bmatrix}
		2 \\ 1
		\end{bmatrix} * \begin{bmatrix}
		2 & 1
		\end{bmatrix} = 5$
		
		\item $\begin{bmatrix}
		2 & 1
		\end{bmatrix} * \begin{bmatrix}
		2 \\ 1
		\end{bmatrix} = \begin{bmatrix}
		4 & 2 \\ 2 & 1
		\end{bmatrix} $
		
		\item $\sqrt{4+1} = \sqrt{5}$
		
		\item $\nabla_{x}(f(x)) = \nabla_{x}(x^{T}Ax)$ \newline
		$= x^{T}*A^{T} + x^{T}A$ \newline
		NOTE: $A^{T} = A$ \newline
		$= 2X^{T}*A = 2*(\begin{bmatrix}
		2 & 1
		\end{bmatrix} * \begin{bmatrix} 1 & 2 \\ 2 & 4 \end{bmatrix}) = \begin{bmatrix}
		 8 \\ 16
		\end{bmatrix}$
		
		\item $\nabla_{x}(2x^{T}A)$ \newline
		$= 2A = \begin{bmatrix}
		2 & 4 \\ 4 & 8
		\end{bmatrix} $
		
		\item 2
		
		\item Var(Y) = E[$Y^{2}$]-$(E[X])^{2}$ \newline
		Vary(Y) + $(E[X])^{2}=E[Y^{2}]$
		
		\item E(y+w) = E(y) + E(w) = 2.7 + 3.1 = 5.8 \newline
		Var(y+w) = Var(w) + Var(Y) - 2Cov(w,y) \newline
		because w,y are independent cov(w,y) = 0 \newline
		Var(y+w) = Var(w) + Var(Y) = 8 + 15 =23
		
		\item The formula for the normalizing constant in this case is :$\frac{1}{(2\pi)^{1}*|\sum|^{1/2}}$ \newline $|\sum|$ is determined by taking the determinant of the covariance matrix: $|\sum|= 12-3=9$  \newline s$ |\sum|^{1/2}=3$ \newline
		$\frac{1}{(2\pi)^{1}*|\sum|^{1/2}}$= $\frac{1}{2\pi*3} = \frac{1}{6\pi}$
		
		\item support(z) = k $\epsilon$ {0,1}
		
		\item By the Binomial Theorem: $\binom{n}{k}p^{k}(1-p)^{n-k}$
		
		\item $h(x_{1}) = 1/3*x_{1}^{3}-1/2*x_{1}^{2}-6x_{1}+27/2$ \newline
		$ h'(x_{1})= x_{1}^{2}-x_{1}-6$ \newline
		$ (x-3)(x+2) = 0$ \newline
		$ x = 3; x=-2 $ \newline
		$h(3) = 1/3*3^{3}-1/2*3^{2}-6*3+27/2 = 0$ \newline
		$h(-2) = 1/3*-2^{3}-1/2*-2^{2}-6*-2+27/2 = 20.833$ \newline
		now check the boudnary:
		$h(-4) = 49/6$ \newline
		$h(4) = 17/6$ \newline
		Thus the max is when $x_{1} = -2$
		
		\item Looking at the values computed in the previous question the min occurs when $x_{1} =  0$
		
		\item $\int_{0}^{1}1/z*h(x) = 1/z*\int_{0}^{1} 1/3h(x)^{3}-1/2(x)^{3}-6x+27/2 = 1/z*[1/12*x^{4}-1/6*x^{3}-3x^{2}+27/2*x]_{0}^{1} = 1/z*12/125$\newline
		$1/z*125/12=1$ \newline
		$z = 125/12$
		
		\item $b(x_{1}*x_{2}^{3})$\newline
		$ \int_{A} b(x)dx = \int_{A} x_{1}*x_{2}^{3}= [9/8*x_{2}^4]_{0}^{2}= 9/8*16=9*2 = 16$
		
		\item $c(x) = x_{1} + \sqrt{3}*x_{2}$\newline
		$x_{1}^2+x_{2}^2 = 1$ \newline
		$x_{1} = \sqrt{1-x_{2}^{2}}$
		$c(x_{2}) = \sqrt{1-x_{2}^{2}} + \sqrt{3}*x_{2}$ \newline
		$\frac{d c(x_{2}}{dx_{2}} = \frac{-x}{\sqrt{1-x_{2}^{2}}}=0$ \newline
		$\sqrt{3}*\sqrt{1-x_{2}^{2}} = x_{2}$ \newline
		$3*(1-x_{2}^{2})=x_{2}$ \newline
		$x_{2} = \sqrt{3/4} \newline x_{1} = 1/2$
		
		\item $g(x) = -x_{1}*log(x_{1}) - x_{2}*log(x_{2}) \newline
		x_{1}=1-x_{2} \newline		g(x_{2})=(1-x_{2})*log(1-x_{2}-x_{2}*log(x_{2})) \newline
		\frac{dg(x_{2})}{dx_{2}}= -(1-x_{2}*-1/(1-x_{2})-log(1-x_{2}))-x_{2}*1/x_{2}-log(x_{2}) \newline=log(1-x_{2})+1-log(x_{2}-1) = log(1-x_{2})-log(x_{2})\newline
		1-x_{2}-x_{2}=0 \newline
		x_{2} = 1/2 => x_{1} = 1/2$
	\end{enumerate}
\end{document}
